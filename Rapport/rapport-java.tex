\documentclass[a4paper, 11pt]{report}

\addtolength{\hoffset}{-1cm}
\addtolength{\textwidth}{2cm}

\usepackage[utf8]{inputenc}
\usepackage[frenchb]{babel}
\usepackage[T1]{fontenc}

\usepackage{tikz}

\begin{document}

\chapter{Client Java - Vincent Botbol}

% Petite explication du fonctionnement - ant build - choix d'implem (swing, java 7)
% Extensions

Ce client a été réalisé en \emph{Java} (version 7). L'interface graphique utilise 
la bibliothèque \emph{Swing}. Les extensions implémentés sont : la discussion instantanée,
les comptes utilisateurs et le mode spectateur. Les tests ont été réalisés sur le serveur OCaml
fourni ainsi que sur un serveur factice java également présent dans l'archive.

Le programme démarre sur une fenêtre de connexion proposant à l'utilisateur d'entrer
l'adresse du serveur, son pseudo, son mot de passe (pour les comptes utilisateurs), de choisir
le mode spectateur, et de se connecter (protocole ``LOGIN'' si le mot de passe est fourni, ``CONNECT'' sinon) ou d'enregister son compte.

Après la réponse du serveur, l'interface graphique du jeu s'affiche. Celle-ci se décompose en
trois parties :
\begin{itemize}
  \item La grille de jeu
  \item Une zone de texte pour les informations émises par le programme et par le serveur
  \item La zone de discussion
\end{itemize}\medskip{}

Pour communiquer avec les autres joueurs (et spectateurs), il suffit d'entrer son texte 
puis de valider avec ``entrée'' ou bien de cliquer sur ``Envoyer''.

La grille de jeu permet les interactions de l'utilisateur et la zone de notification transmet les
informations importantes.

Dès la réception du ``SHIP'', l'utilisateur place son bateau en déplacant la souris sur la grille
de jeu (clique droit pour le tourner), et valide sa position avec clique gauche. Le client
réalise une vérification interne sur le placement avant d'envoyer la commande ``PUTSHIP'' au serveur.
Les bateaux sont représentés par des rectangles de couleur rouge, grise, verte ou jaune selon
leur positions dans la liste de joueurs.

La disposition des bateaux terminée, le client attend le ``YOURTURN'' avant de redonner la main 
à l'utilisateur. Lorsque son tour est arrivé, le drone apparait ainsi qu'une zone d'action basée
sur les mouvements accordés. L'utilisateur finit son tour lorsqu'il ne lui reste plus de mouvement
ou bien qu'il passe son tour (clique droit). En cliquant sur sa position, le joueur active le laser.
Si un bateau a été touché pendant ce tour, l'utilisateur peut le constater grâce à une
croix verte apparaissant à la position du bateau touché. Si le coup est manqué, un rond magenta
est disposé. Si l'un de ses bateaux est touché pendant le tour d'un des adversaires, le client est
notifié d'une croix rouge à l'emplacement du bateau détruit.

En mode spectateur, l'utilisateur ne peut intéragir. Il peut uniquement communiquer par le biais
de la discussion instantanée. Il est notifié des touches de bateaux par un code couleur
propre au joueur ayant effectué l'action.

\section{Implémentation}

\section{Modèlisation}
L'architecture du client suit un patron de conception MVC strict.
Ce schéma illustre la modélisation de l'application :

\begin{tikzpicture}
  \node[draw] (S) at (-7, 0) {Serveur};
  \node[draw] (M) at (0, 4) {Modèle};
  \node[draw] (V) at (4, 0) {Vue};
  \node[draw] (C) at (-4, 0) {Contrôleur};
  \draw[<->,>=latex,dotted] (S) -- (C);
  \draw[->,>=latex] (C) to[bend right] (V);
  \draw[->,>=latex] (C) -- (M)
  node[midway, sloped, above]{Calcul du nouvel état};
  \draw[->,>=latex] (V) |- (M)
  node[midway, above]{Requêtes d'état};
  \draw[->,>=latex,dotted] (V) -- (C)
  node[midway, above, sloped]{\'Evènements utilisateurs};
  \draw[->,>=latex,dotted] (M) -- (V)
  node[midway,sloped, above]{Notifications de m-à-j};
\end{tikzpicture}


On distingue donc trois parties principales :
\begin{itemize}
  \item Le modèle, gérant toute la partie logique de l'application : la grille, le calcul des déplacement, la validité des coups pour la vérification côté client, etc..
  \item La vue, s'occupant d'afficher les composants graphiques pour communiquer avec l'utilisateur.
    Cette dernière va s'enregistrer auprès du modèle qui va notifier la vue lorsqu'elle est censée
    se mettre à jour.
  \item Le contrôleur, connaissant le modèle et la vue, va permettre de connecter l'interface
    graphique (boutons, champs textes, ..) et ainsi de traiter les divers évènements que 
    l'utilisateur est susceptible d'envoyer par le biais de l'interface. En réceptionnant ces
    évènements, le contrôleur établit les calculs à effectuer par le modèle. Le contrôleur 
    s'occupe également de la communication avec le serveur, en traduisant en requêtes
    les actions de l'utilisateur transmises par la vue et en réceptionnant les réponses du
    serveur qui démarreront les calculs.
\end{itemize}

Pour expliciter cet modélisation, on peut détailler un exemple, celui de l'envoi d'un 
message instantané :
\begin{enumerate}
  \item Le client saisi son message dans le champ de texte puis clique sur ``Envoyer''
  \item Le bouton ``Envoyer'' de la vue étant relié au contrôleur, celui-ci procède à la récupération
    du texte au travers de la vue. Puis envoie une requête de type ``TALK/message/'' au
    serveur
  \item Le serveur répond en transmettant la commande ``HEYLISTEN''. Le contrôleur traite
    cette nouvelle commande reçue en appelant le module discussion du modèle qui va
    ajouter ce nouveau message, prévenir la vue qu'il y a un changement de type
    ``discussion instantanée'' et qu'elle doit donc se mettre à jour.
  \item La vue étant enregistrée sur le modèle, elle reçoit cet événement, récupère le nouveau 
    message et enfin l'ajoute à sa zone de discussion.
\end{enumerate}

Cette séquence de communication se repète pour la plupart des interactions du programme.

\section{Concurrence}

Le modèle de concurrence employé est préemptif puisqu'il utilise exclusivement l'API native de Java.
L'interface graphique est exécutée à l'intérieur de l'``Event Dispatcher Thread'' (EDT) 
pour garantir la fluidité de l'interface graphique même lors de calcul long réalisés par le modèle.
Ces calculs sont eux-mêmes exécutés dans des threads\footnote{Fil d'exécution} dédiés et 
extérieurs à l'EDT. Des ``SwingWorker'' ont pour cela été utilisés.

Le principal thread (explicite) de l'application reste la lecture continue des réponses du serveur
incluant donc les différents calculs impliqués.
D'autres threads secondaires sont également déployés. Notamment pour la mise-à-jour de l'interface
graphique ou encore pour effectuer des effets visuels sur la grille de jeu 
(qui ne sont, certes, pas nombreux).


\section{Communications Client-Serveur}

Comme spécifié dans le sujet du projet, la connexion s'appuie sur un protocole TCP sur le port 2012.
Les sockets utilisées sont celles de Java provenant du paquet ``java.net'' 
et les lectures/écritures sont réalisés par canaux tamponés. Les envois et récéptions 
sont textuelles. Les données sont transmises au moyen d'un ``DataOutputStream'' pour augmenter 
la portabilité des données envoyées.
Les lectures sont quant à elles interprétées directement par un ``BufferedReader'' lisant
les commandes reçues au moyen d'un ``readLine()'' (puisque les commandes sont préfixés d'un retour 
à ligne).

Le protocole de communication fourni a été respecté. Le traitement des commandes 
reçues s'effectue par expression régulière. L'utilisation du \emph{look-behind} a permis 
de s'abstraire du cas particulier (tels que le ``\/''). Il a donc suffit de séparer la
chaine de caractère reçue par l'expression ``(?<!\textbackslash{}\textbackslash{})/'' qui peut-être traduite par :
``sépare les champs de la chaine par les slashs présents mais uniquement si celui-ci n'est pas précédé
d'un anti-slash''.

Les slashs saisis par l'utilisateur dans un message instantané, par exemple, sont traités avant 
l'envoi (en les sécurisant par un anti-slash) et de même pour les anti-slashs. La réciproque est 
employée lors de la récéption.

Pour plus de sécurité, les commandes possible à l'envoi sont typés par énumération (ERequest).
Les commandes reçues sont traités et coércées vers une deuxième énumération (EResponse) qui 
générera une exception si la commande reçue ne fait pas partie de l'énumération. 
Le tout est encapsulé dans un objet ``Command'' contenant la commande protocolaire et les arguments
reçus ou à envoyer.

\section{Difficultés}

La principale difficulté de ce projet a été d'établir la structure principale de l'application.
Une fois cette phase réalisée, l'enchainement de l'implémentation des différentes parties
du jeu fut triviale.

Le temps accordé a également été une problématique. J'aurais souhaité améliorer l'aspect visuel 
et sonore de l'application, ajouter quelques extensions et peaufiner l'implémentation 
mais la date limite approchant, je n'ai pas eu cette occasion.

\end{document}
