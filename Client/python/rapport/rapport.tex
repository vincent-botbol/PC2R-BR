\documentclass[a4paper, 11pt]{report}

\addtolength{\hoffset}{-1cm}
\addtolength{\textwidth}{2cm}

\usepackage[utf8]{inputenc}
\usepackage[french]{babel}
\usepackage[T1]{fontenc}


\begin{document}

\chapter{Client Python - Matthieu Dien}

\section{Choix Techniques}

Pour ce client, le choix du langage d'implémentation a été Python car c'est un langage haut niveau, simple, 
avec une librairie satndard très fournie et pour une majorité de librairie graphique implémente
un binding. La librairie graphique choisie est wxPython, s'appuyant sur wxWidgets (écrite en C++), disponible 
sur la plupart des plateformes.
Ce choix a été fait pour simplifier la conception de l'interface graphique, wxPython étant simple d'utilisation et 
très complète par les widgets qu'elles proposent. Ce choix Python/wxPython garantit la portabilité du client.

\section{Implémentation}

Le client se présente sous la forme d'un fichier \emph{main.py}, de deux dossiers 
\emph{connection} et \emph{client}, ainsi que de deux fichiers
\emph{mybuttons.py} et \emph{mysocket.py}.

\subsection{Socket}
Dans \emph{mysocket.py} on trouve la fonction \emph{readline} 
qui permet la lecture sur socket. Cette fonction utilise le mécanisme de ``generator'' python pour retourner ligne par ligne 
les lectures faites sur le socket. C'est une petite surcouche au socket fourni par la librairie standard python qui ne permettait pas cette 
lecture ligne à ligne.

\subsection{Bouton}
Dans \emph{mybuttons.py} on trouve une surcouche au ``BitmapButton'' qui permet la gestion des différents types de 
case que l'on peut rencontrer durant la partie en utilisant un système de ``flag'' binaire pour pouvoir ajouter facilement des caractéristiques aux cases.
Par exemple, une case peut être avec un sous-marin, touché et rouge (car sous le drône).

\subsection{Client}
Le principal du code se trouve dans le dossier \emph{client}, notamment dans le fichier \emph{controler.py}.
On y trouve une classe prinicpale \emph{Controler} qui gère l'interface graphique (UI) et procède à tous les traitements graphiques ainsi que trois autres classes : 
\emph{WaitPlayers}, \emph{PutShip} et \emph{Action}. Chacune de ces classes hérite de \emph{threading.Thread}, équivalent Java de 
\emph{Thread}, ce qui permet de ne pas bloquer l'UI lors de l'attente d'événements tels que la réponse du serveur ou l'appui d'un bouton. 


